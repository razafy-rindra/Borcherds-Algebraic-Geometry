%! TEX root = ./Borcherds_Alg_geo_2.tex
\input{../../template_lectures.tex}
\author{Razafy `Hagamena' Rindra}
\title{Lecture notes: Borcherds' Algebraic Geometry II}
\begin{document}
\maketitle
\tableofcontents
\lecture{Sheaves}
   \section{Introduction}
   Basics of schemes theory, loosely following chap 2 of Hartshorne.
\begin{remark}
    Recall, affine varieties over a field $k$ correspons to commutative rings with the following properties: \begin{enumerate}[label = (\arabic*)]
    \item The ring is an algebra over $k$
    \item It is finitely generated as an algebra
    \item It has no nilpotent elements.
    \end{enumerate}
\end{remark}
We will now look at three examples showing us why we don't want to have these properties:

\begin{example}
    We might look at the coordinate ring of a line, $k[x]$, the behaviour of this is similar to the behaviour of the integers or algebraic number fields like $\Z[i]$, we want algebraic geometry to include these as examples but they are not algebras over a field.
    So maybe we should just delete the first condition.
\end{example}

\begin{example}[]
We might want to look at $k(x)\supseteq k[x]$, the field of all rational functions on the line. And at  ${k(x)}_{(x)} = \{\frac{p}{q}\mid q(0)\neq 0\}$. But these fields are not finitely generated over $k$. So we might want to delete the second statement.
\end{example}

\begin{example}[]
    If are looking at Bézout theorem, we might intersect two varieties, for example the varieties:
    \begin{equation*}
    y=x^2 \ \text{and} \  y=0
\end{equation*}
If we want to take the intersection of two varities we quotient out by the ideal generated by both varieties, in this case:
\begin{equation*}
    k[x,y]/(y,x^2-y) \simeq k[x]/(x^2)
\end{equation*}
In classical geometry we would simplify as $k[x]/(x) \simeq k$, but we are losing information when we do this. In this case we are forgetting that the intersection is a double point. The ring $k[x]/(x^2)$, keeps track of that, the only disadvantage is that this ring has nilpotents. So it does seem like we should drop the third condition.
\end{example}
What are we left with? \textbf{all commutative rings}. We will now define \textbf{affine schemes} which will correspond to all commutative rings.

In order to study these we need to learn about Sheaves.

    \subsection{What are Sheaves?}
    Sheaves were introduced by Loray in $1950$, then were used by Serre and Cartan to study algebraic varieties. Serre introduced them to Alg geo in his paper \textit{Faisceaux ALgébriques Coherents}. This paper is still possibly the best introduction to Sheaves in algebraic geometry.

\paragraph{Motivation}
In algebraic geometry before the introduction of sheaves people found all sorts of invariants. For e.g. Arithmetic genus, people studied surface embedded into three-space and found that the arithmetic genus didn't depend on how it is embedded. Unfortunately this is not easy to define or calculate. One way to define it is \[
    p_a = binom{\mu_0 -1}{2} - (\mu_0-4)\epsilon_0+\frac{\epsilon_1}{2}-\epsilon_0+2t
\]
$\mu_0 $ is the degree of the section of the surface by a hyperplane and $\epsilon_i$ and  $t$ are other invariants of the surface.

People found by brute force that this combination didn't change if we changed the embedding of the surface, but didn't know how to generalise this into $3$ or more dimensions.
If we introduce sheaves it turns out that the arithmetic genus is easier to define for surfaces the arithmetic genus is \[
    p_a = \chi - 1.\]
Where $\chi$ is the euler characteristic: \[
    \chi = \dim H^0(x) - \dim H^1(x)+ \dim H^2(x).\]

\

There is another kind of genus, called the geometric genus. Both the arithmetic and geometric genus were definied analogously to the genus of a riemann surface. The \textit{irregularity} is another invariant related to the difference between these genus. It was not clear how to generalise it to higher dimensions, but using cohomology of sheaves we see that \[
   \textit{``Irregularity''}=\dim H^1.\]
 We can generalise this to higher dimensions by looking at higher dimensions of the cohomology groups of sheaves.


\begin{example}[]
    Suppose $X$ is a topological space with open set $U$. We define a (pre)-sheaf by letting  \[
    F(U) = \text{continuous functions from }U\rightarrow \R.\]
    So $F$ is a map from the open sets to \textbf{ableian groups}. It has the following properties:
    \begin{properties}
    \item If $V\subseteq U$, with  $V$ and  $U$ open, then there is a restriction map $\rho_{UV}\colon F(U)\rightarrow F(V)$
    \item We have, for $W\subseteq V\subseteq U$ with $W,U,V$ open \[
 \rho_{UW} = \rho_{UV}\rho(UV) \ \text{and} \ \rho_{UU} = \text{identity}
\] 
    \end{properties}
\end{example}

This example motivates the following definition.

\begin{definition}
A \textbf{presheaf} (of abelian groups) is a map: \[
    F\colon \text{open sets}\rightarrow \text{ablelian groups}
.\]
And maps $\rho_{UV}\colon F(U)\rightarrow F(V)$ satisfying the properties in the example.
\end{definition}


Another way of thinking of sheaves is to think of $F$ as the \textit{contravariant functor} from the category of open sets of $X$ to the category of abelian groups. What does this mean?

\begin{definition}
    The category of open sets of $X$, is the category whose objects are the open sets of $X$. And the morphisms:
    \[ 
    U\rightarrow V\colon
    \begin{cases}
    1 \ \text{if} \ U\subseteq V\\
    0 \ \text{otherwise}
    \end{cases}.\]
\end{definition}

Now the contravariant functior $F$ is just assigning an abelian group to each open set, together with a map from the abelian group of $V$ to the abelian group of $U$ if $V\suibseteq U$.
While this definition seems abstract, this suggests some powerful generalisation of a sheaf if we replace the category of open sets by any other category, and likewise we can replace the category of abelian groups by anothercategory.

We will not use this abstract approach to presheaves in these introductory lectures.

\subparagraph{More examples of presheaves}
\begin{itemize}
\item \begin{example}
    Let $X = $ smooth manifold, and for each open set $U$, we take $F(U) = $ smooth vector fields on $U$.
\end{example}  
\item \begin{example}
  $X =$ Affine variety, and $F(U) = $ regular functions on $U$.
\end{example} 
\item \begin{example}
  $X =$ Riemann surface, and $F(U) = $ holomorphic functions on $U$
\end{example}
\item \begin{example}
    Let $F(U) = A$, where $A$ is a \textbf{fixed abelian group}. This is a presheaf for trivial reasons. But not according to Hartshorne's definition, since he puts in the condition that $F(\emptyset) = $ zero group.
\end{example}
\end{itemize}

\paragraph{But what is a sheaf?}
Except for example $8$ our example satisfy an extra condition. 
Most of our examples have some sort of local property, they are smooth iff they are smooth locally everywhere and a sheave captures this condition. 


\begin{definition}[]
    A \textbf{sheaf} is a presheaf satisfying the following properties: Suppose $U = \bigcup U_i$ then 
\begin{properties}
    \item $f\in F(U)$, if the image of $f$, by the map $\rho_{UU_i}$ in all $U_i$ is zero, then $f=0$. A sheaf satisfying this condition is said to be \textbf{seperated}. It says that a function is defined by it's values locally everywhere. We can't have a function that is zero on $U_i$ but mysteriously non-zero on the whole set.
    \item Given $f_i\in F(U_i)$,  and $f_j\in F(U_j)$ that have the same image in $F(U_i\cap U_j)$ for all  $i,j$. Then we can find $f\in F(U)$, whose image is $f_i$ in $F(U_i)$.
\end{properties}
\end{definition}

\subparagraph{Presheaf that is not a sheaf}%
\label{subp:PresheafNotSheaf}
The above example example $8$, if $F(U) = A$, where  $A$ is fixed an  $A\neq 0$. This is not a sheaf since a sheaf has the property that $F(\emptyset) = 0$, since the empty set can be covered by the empty collection of open sets.

But what if we take
\begin{equation*}
    F(U) = \begin{cases}
    A \  \text{if} \ U\neq \emptyset\\
    0  \ \text{otherwise}
\end{cases}?
\end{equation*}

This is still not a sheaf, if we take $U = U_1\cap U_2$ with  $U_1\cap U_2 = \emptyset$. Then we have  $F(U) = A$, however we have  $F(U) = F(U_1)\times F(U_2) = A\times A$, if  $F$ is a sheaf which is not true in general.

\section{Étalé spaces}

\begin{example}[]
    Let us look at $\C[x]$: 
\begin{itemize}
    \item The affine line $\C$ corresponds to the maximal ideal of this ring (recall $\alpha$ corresponds to  $(x-\alpha)$).
    \item The Zariski topology, is where the open sets have a basis of sets $D_f$ which is the set of points where $f\neq 0$, for some polynomial $f$. 
    \item The regular functions on each open set: $F(D_f) = \{\text{rational functions of the form} \frac{g}{f^n}\}$
\end{itemize}
    
    

    \

    We do the same for the ring of integers $\mathbb{Z}$. 

\begin{itemize}
    \item Instead of the affine line, we look at the set of maximal ideals called $spec_M(\Z)$.
    \item For $f$ an integer, we define open sets $D_f$ where $D_f$ is the set of primes, $p$ such that $f\not\in (p)$. i.e. $p\not\mid f$.
    \item We define $F(D_f) = \{\text{set of rational numbers } \frac{g}{f^n} \}$. 
\end{itemize}

So we are sort of pretending that an integer is a function on $spec_M(\Z)$, but notice that $f\in \C[x]$ has values at all $x\in \C = \C[x]/(x-\alpha)$. But for $f\in \Z$, it takes values in the field $\Z/(p)$, so the function takes values in a field that varies with the point. 

We also see that $F$ satisfies the sheave property, if we take the open sets $U_1 = \{5,7,\cdots\}$ and $U_2 = \{3,7,\cdots\}$.

So \begin{align*}
    F(U_1) &= \left\{ \frac{a}{2^n3^m} \right\}\\
    F(U_2) &= \left\{ \frac{b}{2^n5^m} \right\}\\   
\end{align*}

We have $F(U_1\cup U_2) = \{ \frac{c}{2^n} \}$, we see that if we have an element in $F(U_1)$ equal to an element in $F(U_2)$ i.e. $\frac{b}{2^n3^m} = \frac{a}{2^n5^m}$, then by fundamental theorem of arithmetic we know that it is actually of the form $\frac{c}{2^n}$.

The point of this is that we can study the ring $\Z$ geometrically!
\end{example}

\begin{remark}
 A basic theme of sheaves is that: Sheaves of sets are similar to sets, and Sheaves of abelian groups are similar to abelian groups.


 In particular we can form categories of sheaves of sets and a categories of sheaves of abelian groups. Where the morphisms from $F$ to $G$, is such that for each open set  $U$ there is a morphism $F(U)\rightarrow G(U)$, that is compatible with the restrictions maps as so:

  \begin{equation}
 \begin{tikzcd}
   &  F(U) \ar[r] \ar[d,"\rho"] & G(U) \ar[d, 
    "\rho"] \\
   V\subseteq U: & F(V) \ar[r] & G(V)
 \end{tikzcd}
 \end{equation}


 This makes sheaves into a category, and the category of sheaves of sets is a weak model of set theory called a \textbf{topos}.

 Similarly the category of sheaves of abelian groups is very similar to the category of abelian groups.
\end{remark}


\subsection{What about exact sequences?}
 We might have an exact sequence, of abelian groups: \[
    0 \rightarrow A \rightarrow B\rightarrow C \rightarrow 0
.\]
What does it mean for a sequence of sheaves to be exact? The obvious definition is that \[
    0 \rightarrow A(U) \rightarrow B(U)\rightarrow C(U) \rightarrow 0 \ \text{is exact for all} \ U
.\]

But this definition is \textcolor{red}{\textbf{WRONG!}} Understanding why this is wrong is one of the fundamental things in sheave theory. We will look at an example to see why this is wrong:

\begin{example}[]
    Let $X_1 = X$ be a circle, and $X_2$ is a circle that wraps around itself twice. Now we define a sheaf:  \[
    F_2(U) = \ \text{sections }U\rightarrow X_2
 .\]
And \[
 F_1(U) = \ \text{sections }U\rightarrow X_1    
.\]
 
We get a natural map $F_2\rightarrow F_1$. The map $X_2\rightarrow X_1$ is surjective but  $F_2(X)\rightarrow F_1(X)$ is NOT surjective. Notice that$F_1(X)$ is a point, but $F_2(X)$ is empty. We have two different definitions of surjective, $X_2\rightarrow X_1$ is a sort of local definition of being surjective, but $F_2(X)\rightarrow F_1(X)$ is a sort of global definiton.
\end{example}


Suppose we have any continuous map $A\rightarrow X$, then we can let $F(U) = $ continuous sections $U\rightarrow A$, i.e.\ maps from $U$ to $A$ such that if you compose it with the projection from $A\rightarrow X$ we get the identity. This is always a sheaf, so we can construct a sheaf from any map from $A$ to $X$.

Similarly if we have maps \begin{equation}
\begin{tikzcd}
    A \ar[r] \ar[d] & B \ar[dl]\\
    X &
\end{tikzcd}
\end{equation}
That commute, and $F$ is a sheaf of $A$, and $G$ is a sheaf of $B$, then there is an induced map $F\rightarrow G$.

\

Now $A\rightarrow B$ might be onto, but $F(X)\rightarrow G(X)$ might not be. We have the following problem: Does a sheaf $F$ come from a map $A\rightarrow X$ for some $A$? It turns out there is a nice way to construct $A$.

 \paragraph{Étalé space of a presheaf $F$}

\begin{definition}[]
    We want to construct $A\rightarrow X$. Suppose we have $x\in X$, we want to know what is the fiber of $A$ over $X$? The point of the fiber is given by a section $f\in F(U)$ for some nbh $U$ of $p\in U$. 
    We need to say when these are the same. $f,g$ are the same point of the fiber if `$f,g$ are the same near $x$`. 
    This means that the image of $f,g$ in $V$ are the same for small $V$, with $p\in V$.
If we think the presheaf as continuous sections of something the fiber is roughly equivalence classes of sections, where they are equivalent if they are the same near $p$. 

\

We now must create a topology, suppose we are given $f\in F(U)$, we form an open set: for each  $p\in U$ we take the image of $f$ in the fiber over $p$. 
The union of all these images will be an open set. Open set of this form will form a base of the topology.

This gives us our $A\rightarrow X$. We call $A$ the \textbf{Étalé space} of the sheaf $F$. 

We say a map from $A$ to $X$ is \textbf{Étale} if for all  $a\in A$ there is a nbh $V$ of $A$ such that $V\rightarrow$ image in $X$ is a homeomorphism. 
\end{definition}

\section{Exactness and sheaves}


\end{document}
